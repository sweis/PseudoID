\documentclass{llncs}

\usepackage{times}
\usepackage{graphicx}
\usepackage{url}
\usepackage{verbatim}
%\usepackage{fullpage}

\title{PseudoID: Enhancing Privacy for Federated Login}
\author{Arkajit Dey\inst{1} and Stephen Weis\inst{2}}
\institute{Massachusetts Institute of Technology, Cambridge, MA, USA 02139
\and
Google Inc., Mountain View, CA, USA 94043}

\begin{document}

\maketitle

\begin{abstract}
This paper proposes a privacy-preserving federated login system named
\emph{PseudoID}. PseudoID enhances individual user privacy and
protects identity providers from being compelled to reveal user
data. This system is based on blind digital signatures and is
compatible with a popular federated login system named OpenID. We
propose several extensions and discuss some of the practical
challenges that must be overcome to further protect user privacy.
\end{abstract}

\section{Introduction}
\label{sec:intro}

Internet users often manage login credentials for many accounts across
multiple web sites. This is both an inconvenience and a potential
security risk, as users often resort to reusing passwords. Users also
become accustomed to typing usernames and passwords in many different
interfaces, making them more susceptible to phishing, that is, having
their credentials stolen by spoofed websites.

This has led to several efforts to create web single sign-on (SSO)
systems. One SSO model is for users to have a single \emph{identity
  provider} (IDP) that will be used for all logins. Arbitrary web
sites may then become \emph{relying parties} (RPs), who delegate
logins to the IDP. The IDP handles authenticating the user and
attesting an identity back to the RP.

Some proposals, such as Microsoft Passport (now Windows Live ID)
\cite{MSPass} or Facebook Connect \cite{FBConnect}, relied on a
centralized IDP. Other systems, such as OpenID \cite{OID}, allow users
to have identities from among a federation of IDPs. Federated login
systems like OpenID offer more flexibility to end users, since they
are able to choose among many identity providers. Large webmail
providers like Yahoo, Google, and MSN have all adopted OpenID
\cite{YOP,Sac08,WLOP08} and are capable of serving as identity
providers for hundreds of millions of users.

While federated login systems like OpenID have the prospect of making
logins less cumbersome, they may negatively impact user privacy. The
core problem in both centralized and federated login systems is that
all of a user's logins to relying web sites must flow through their
identity provider. A user's IDP can easily link together the various
websites that the user visits. An abusive IDP could, for example, sell
a list of all the RP's that their users logged into without their
consent.

In a federated system, users could avoid identity providers
that abused privacy and choose to use reputable firms. Unfortunately,
honest identity providers may still be compromised and leak logs, or
otherwise be compelled to reveal logs.

Besides simply revealing which sites a user visits, IDPs often reveal
personal information about users through extensions like OpenID
attribute exchange (AX) \cite{AX} or simple registration (SREG)
\cite{Sreg}. The goal of this exchange is typically to pass
information like an email address, real name, or birthdate from an
identity provider to a web site. Automatically obtaining these data
can greatly streamline the user sign-up process for relying parties.

Although most IDPs will prompt users whether they want to reveal this
information, they could reveal whichever data they want to a relying
party. Thus, there is no way for a user to \emph{selectively disclose}
certain properties (e.g. age, gender, etc.) about themselves to an
RP. Much work has gone into developing cryptographic schemes for
selective disclosure \cite{CaLy01,CaLy04,CHL05,CaGr08}, but these have
yet to be widely adopted in practice.

Finally, the user experience of being redirected back and forth
between the RP and the IDP is less than ideal and especially confusing
to new users. It interferes with the user's well-ingrained mental
model of how they interact with websites.

In this paper, we outline a privacy-preserving federated login system
called PseudoID and offer a sample implementation as an anonymized
OpenID provider. The system utilizes blind signatures \cite{Cha82} and
zero-knowledge proofs \cite{GMR89} to allow users to anonymize
themselves before logging onto RPs. The system exposes a blind
signature service that allows users to generate a digital pseudonym
under which they can logon to a RP, but which cannot be linked back to
their true identity. Additional services are provided to support
selective disclosure of user properties.

\section{Federated Login Overview}
\label{sec:fedlogin}

Web users who want to use a particular website most often authenticate
themselves directly to the site by entering a username and
password. This type of basic web login system has both advantages and
disadvantages. Websites have more flexibility in the design of their
login system, but must carry the burden of authenticating users. These
authentication tasks can be needlessly replicated across many
websites. And, account creation can be a major barrier in signing up new
users. Maintaining many sets of user credentials across different sites also
carries a burden for the user, which can lead to password reuse.
%AD: removed esachs reference for login barrier since I couldn't find a specific
%citation. I feel like it's general enough to not need citation. But if we find
%a good citation later on, we can add it.


%The output of the authentication process is a unique
%identifer such as the username with which the website refers to the
%user in subsequent interactions with him.

\begin{figure}
  \centering
  \includegraphics[scale=0.5]{figs/fig-passwd-color.pdf}
  \caption{A familiar web login system where users log into websites
    by entering site-specific credentials.}
  \label{fig:passwd}
\end{figure}

Federated login systems, on the other hand, extract authentication as
a service in its own right. Just as websites rely on third-party services
for traffic analysis, CAPTCHA verification, or file hosting, they can
also rely on separate services for authentication.

Federated login adds a third party to the interaction between the user
and the website: the \emph{identity provider} (IDP). Instead of
authenticating herself to the website directly, the user authenticates
herself to the IDP. The IDP then returns a user identifier to the
website. Thus, the website is often referred to as the \emph{relying
  party} (RP) since it relies on the identity provider for
authentication.

\begin{figure}
  \centering
  \includegraphics[scale=0.5]{figs/fig-fedlog-color.pdf}
  \caption{A federated login system: the relying party (RP) redirects
  the user to her identity provider (IDP) for authentication.}
  \label{fig:fedlog}
\end{figure}

%The user wishes to use a website which \emph{relies} on the identity
%provider to authenticate the user. Instead of traditional login
%systems where the user authenticates himself to the website directly,
%in federated login systems, the user authenticates himself to the
%identity provider instead. Once the IDP has successfully
%authenticated him, it provides an identity for the user to the RP.
%The RP can then refer to the user through this identifier just as
%they would refer to the user through a username or other handle in
%traditional login systems.

Federated login alleviates the need for websites to store user credentials,
making them less desirable targets for hackers who want to hijack user accounts.
The user benefits from federated login too, as they no longer have to manage
separate login credentials for every website they want to use, and can instead
log into a single IDP. Systems like Facebook Connect and OpenID 2.0 are able to
offer one-click logins for relying parties, which greatly simplifies the login
process. For example, Plaxo, a social networking and address book site,
performed a two-click OpenID login experiment where 92\% of users successfully
completed registration after starting the signup process \cite{Ki09}. (The user
had to click a second time to authorize her identity provider to release her
contact information to Plaxo.)

Accordingly, federated login systems are being adopted by a growing
number of internet sites, particularly by large webmail providers and
social networks.  Several different federated login technologies have
arisen over the years, such as Microsoft Passport (now Windows Live
ID), OpenID, Facebook Connect, and SAML.

%TODO(arkajit): talk more about these systems.

However, popular federated login systems have generally been designed
without privacy as a primary concern. Subsequently, there are risks
with widely-used federated login systems which could put sensitive
user data at risk. The problem of user privacy is indeed magnified in
federated login systems since identity providers act as stewards of
user data for multiple websites. This not only makes them more
appealing targets to attackers, but also more acutely responsible for
safeguarding user privacy.


\subsection{OpenID: A web-based federated login system}

One popular example of a federated login system is OpenID
\cite{OID}. Users can claim identifiers in the form of URIs that will
be attested by their identity provider. To login to a website that
supports OpenID, the user enters their OpenID URI and the RP redirects
them to their identity provider's page. The IDP authenticates the user
through whatever means (e.g. passwords, smart cards, etc.) it desires
and then returns the user to the RP with either a positive or negative
assertion that the user owns the claimed identitfier. If the RP
receives a positive assertion from the IDP, it may allow the user to
enter the site under the name of the claimed identifier.

With the advent of OpenID 2.0 \cite{OID2}, the protocol also began to support
the concept of \emph{directed identity} \cite{Cam06} or \emph{private digital
addresses} through a new feature called \emph{identifier select} \cite{RR06}.
This allows the user to just specify the URI of his identity provider instead
of claiming a personal identifier when logging into a website. The site then
redirects the user to the IDP as before, but the IDP now has the opportunity to
select an identifier for the user. Upon successfully authenticating the user,
the IDP returns the selected identifier to the site.

This allows the IDP more flexibility in selecting identifiers for its users. For
example, the IDP may decide to return a different identifier for the same user
for different RPs in order to implement true directed identity as defined in Kim
Cameron's Laws of Identity \cite{Cam06}. And indeed, some OpenID providers like
Google do return a per site unique identifier rather than a globally unique
identifier for its users.

\subsection{Privacy Concerns in Federated Login}

In federated login systems, users entrust identity providers to manage their
identity, so privacy concerns may seem relatively minor. After all, in OpenID
and most other major single sign-on systems, a malicious identity provider could
easily impersonate users to relying parties. However, even if an identity
provider is not corrupt, there are privacy concerns for
\emph{honest, but retentive} providers who may be forced to reveal log data.

The core privacy issue with widely-deployed federated login systems are that
a user's identity can be correlated with the sites they log into. For example,
OpenID identity providers will authenticate users, then redirect them directly
back to a relying party. This makes it trivial to track all the relying parties
a user logs into. This is the same case for Live ID or Facebook Connect; the
identity provider knows all the sites a user logs into.

One might develop a different federated login flow where a user acted as an
intermediary between relying parties and identity providers. The user could
avoid passing any information about the specific relying party to the identity
provider. In this case, the identity provider might return an anonymized
identifier via the user to the relying party. However, if the provider colluded
with the relying party, they could link the user's real identity with the
account on the RP.

An identity provider could abstain from logging, could try to
anonymize or delete identifying information, or could simply destroy
logs completely. Logs are retained for many reasons including
analytics, diagnostics, and security auditing. Abstaining from logging
is generally not a viable option. 

Removing or anonymizing identifying information ex post facto is one
option, but has proved difficult in practice. Supposedly anonymized
logs released by AOL \cite{BarZel06} and Netflix \cite{NaSh08} were
both de-anonymized to some extent. An identity provider would need to
be vigilant and thoroughly scrub logs. They would also need to ensure
that identifying data was not being logged by an unrelated service. 

An added issue is that both logs anonymization and destruction 
may be subject to data retention laws that specify minimum retention
periods \cite{EUDir}. IDPs may be legally compelled to collect
identifying data and retain it for a minimum period.

Given these considerations, we will focus on privacy in a setting where:
(1) identity providers are unable to assure that logs are not retained, and
(2) identity providers may be compelled to reveal logs at some time.

There are several real-world risks where \emph{honest, but retentive}
identity providers may threaten user privacy. One risk is simply if
the provider were compromised and logs were leaked by an
attacker. Another risk is for providers operating in jurisdictions
where logs may be seized without due legal process. 

Because of these risks, identity providers may have an interest in \emph{not}
being able to link a particular user's identity to logins on a particular
relying party. An identity provider may want to provably show that
they cannot link logins on a relying party to a particular user. With
this in mind, we informally define what it means for an idenity
provider to be \emph{private} in Section
\ref{sec:private-fed-login}. Section \ref{sec:pseudoid} propose a
practical system that meets this definition.

\section{What is Private Federated Login?}
\label{sec:private-fed-login}

Both ``privacy'' and ``federated login'' are loaded terms that carry
different meanings in different contexts. For the scope of this paper,
we are going to focus on federated login systems with an OpenID-like
login flow illustrated in Figure \ref{fig:fedlog}. The relevancy to
other types of federated login systmes may vary.

We will now define several informal properties of OpenID-like
federated login systems may or may not satisfy. We then will contrast
which objectives are satisfied by the status quo and several ``straw
men'' proposals. Section \ref{sec:pseudoid} will present
\emph{PseuoID}, which aims to meet all the following properties.


\begin{definition}[Consistency]
\label{def:consistency}
Users may present credentials to an identity provider such that a 
consistent identifier will be returned to relying parties over
multiple sessions.
\end{definition}

\begin{definition}[Unlinkability]
\label{def:unlinkability}
% TODO(sweis)

\end{definition}

\begin{definition}[Abuse-Resistance]
\label{def:abuse}
% TODO(sweis)
\end{definition}


%In addition to preventing tracking by the identity provider, there are several
%other qualities that a privacy-preserving federated login system should
%have. The system should still allow the user to login consistently under her
%selected alias. That is, subsequent visits to a website by a user whose nickname
%is ``Alice'' should be able to be correlated. Further, the system should permit
%the user to control how her nicknames are linked across various relying parties.
%For example, she should be able to decide whether her identity on her bank site
%and her hospital site should be the same (linkable) or different (unlinkable).

%While maintaining these properties, the system should also minimize its trusted
%components, be reasonably resillient against spammers, and be applicable to any
%type of federated login system.

A \emph{private federated login} system should try to satisfy the following
goals:

\begin{enumerate}
  \item \textbf{Consistent Login}: A user should be able to correlate multiple
  visits to a website with the same alias.
  \item \textbf{User-Controlled Linkability}: The user alone, and not her IDP or
  any RPs, should be able to decide whether any two of her aliases on separate
  RPs should be linked together.
  \item \textbf{Minimal Trust}: The system should minimize its trusted
  components. That is, even a malicious IDP or a set of colluding RPs, should
  not be able to violate the system's guarantees for the user's privacy.
  \item \textbf{Cheater Resillient}: The system should incorporate some
  protections against abusive users such as spammers.
  \item \textbf{Usability}: The system should be as easy to use as possible and
  try to prevent the user from making any catastrophic errors such as
  inadvertently revealing her identity.
\end{enumerate}

%TODO(arkajit): add a point about selective disclosure here? need to flesh that
%out a bit more...

\subsection{The ``Yes'' Identity Provider}

The simplest identity provider is the one who simply authenticates any user who
asks with a random identifer. While this is ideal for one-time anonymous logins
to websites, it is really impractical for most users since it fails to offer
them a consistent identity with which to login. And relying parties suffer as
well since they accumulate many one-time use accounts that they cannot correlate
which severely limits their ability to provide a useful service to the user.
Needless to say, users cannot link identities across multiple websites either.
Further, such a system imposes no restrictions on would-be spammers and thus
violates our fourth criterion too.

\subsection{The Random Identity Provider}

A slight improvement on the ``Yes'' identity provider is the random IDP which
does the following. The first time a user visits, it tags them with a random
identifier and returns it to the relying party. On subsequent visits, the user
presents his assigned identifier and the identity provider asserts its
authenticity to the website.

This does allow the user to create a consistent persona across multiple RPs, but
also allows the various RPs to collude to piece together the user's browsing
history. The IDP, meanwhile, ceases to provide any true service as the user is
allowed to generate his own arbitrary nicknames to login. In particular, there
are no protections against two users picking the same nickname and effectively
sharing an account on the RP.

\subsection{No Identity Provider}

Finally, the degenerate case is where there is no identity provider at all, i.e.
the user resorts to the simple traditional per-site login. This is the status
quo, where the user creates different nicknames (usernames or handles) at each
website and must remember passwords manually or through a password manager. If
the user wants to link his identities on two websites, he may decide to choose
the same nickname on both. But since each site has its own namespace of
usernames, there is no guarantee that the user ``Alice'' on one site is the same
as the user ``Alice'' on another site. Moreover, this presents a terrible user
experience as the user is prone to making privacy-compromising errors. Overall,
such a setup is not a system at all, is very brittle, and not extensible towards
new features such as selective disclosure of user attributes to relying parties.

\section{PseudoID: A privacy-preserving federated login system}
\label{sec:pseudoid}
%TODO(arkajit): eliminate abbreviations in prose, OK for diagrams

PseudoID consists of three components: a blind signature service (BSS) or blind
signer, an attribute authentication service (AAS), and a private identiy
provider (IDP). The login flow with PseudoID consists of a one-time setup phase
and then a simplified sigon phase whenever the user needs to authenticate to her
identity provider.

During the setup phase, the user logs on to the blind signer using a normal
authentication scheme, such as entering a username and password. The user may
ask the blind signer to generate an access token binded to a desired nickname
and for use at a particular private identity provider.

This access token then allows the user to login to the identity provider
pseudonymously under their chosen nickname. The system requires that the
identity provider trust the blind signer to authenticate users on its behalf.
The identity provider in turn offers authentication services for other relying
parties that the user wishes to use.

In practice, the blind signer and the identity provider may even be the same
entity or at least closely linked (e.g. bss.google.com and idp.google.com).

The first subsection reviews blind signatures, the cryptographic primitive used
to guarantee the user's anonymity. The second and third subsections elaborate on
the functions of the blind signer and the private identity provider. The later
sections show how PseudoID can be integrated with OpenID, detail some necessary
assumptions and discuss the performance costs.

The AAS is part of on-going and future work that will be discussed in a later
section.

\subsection{Blind Signatures}

Traditional public key digital signature schemes \cite{DH76} consist of a
private signing function $\sigma$ known only to the signer and a public verifying
predicate $V$. Then for any message $m$ that is provided to the signer to be
signed, a verifier can check that $V(m, \sigma(m))$ is true.

Blind signature systems \cite{Cha82} augment this traditional scheme with a
blinding function $B$ and its inverse unblinding function $B^{-1}$, such that
$B^{-1}(\sigma(B(m)) = \sigma(m)$ and both functions are known only to the
provider.

The provider wishes to obtain a signature $\sigma(m)$ on some message $m$ without
revealing the contents of $m$ to the signer. Thus, the provider sends the
blinded message, $B(m)$, to the signer. The blinded message leaks no information
about $m$ to the signer. The signer then signs the blinded message and returns
$\sigma(B(m))$ to the provider. Finally, the provider unblinds this signed
message to obtain
$$B^{-1}(\sigma(B(m))) = \sigma(m),$$
a valid signature on $m$ that can be publicly verified.

One example of a blind signature system is Chaum's RSA blind signatures. In a
standard RSA digital signature system, the public parameters are a modulus $n$
and an exponent $e$. Only the signer knows the private exponent $d$.

To blind a message $m$ prior to sending it to the signer, the provider
multiplies it by a random blinding factor $r$ to produce $B(m) = mr^e$. The
signer signs $B(m)$ to produce
$$m^d r^{ed} \equiv m^d r \pmod n$$
by Euler's theorem. Since the provider can compute $r^{-1}$, he can unblind the
returned signature to obtain
$$m^d \pmod n,$$
a valid signature on the original message $m$.

%TODO(arkajit): cite more blind sig scheme examples like El Gamal

\subsection{Blind Signature Service}

PseudoID contains a blind signature service (BSS) or blind signer that allows
users to login normally and generate access tokens with which they can login
anonymously to RPs. This setup phase is outlined in Figure \ref{fig:bss-setup}.

Upon visiting the blind signer, the user first authenticates itself to the
service perhaps through a traditional password-based authentication scheme. Then
she may select a nickname under which she wants to be able to login to an RP.
The nickname is bundled up into an access token which the BSS will sign to
attest its validity.

To prevent the signer from being able to associate the user with her nickname,
the user first blinds the token before sending it to the service. And upon
receiving the singed token back from the service, the user unblinds it to obtain
a still validly signed token that the signer has never seen before.

\begin{figure}
  \centering
  \includegraphics[scale=0.6]{figs/fig-bss-setup-color.pdf}
  \caption{Blind Signer Setup: (1) User first authenticates herself to the BSS
  normally. (2) Then the user sends the BSS a blind token to sign. (3) The BSS
  signs the token and returns it. (4) The user unblinds the blind signed token
  to obtain a valid, untraceable access token (AT).}
  \label{fig:bss-setup}
\end{figure}

\subsection{Identity Provider}

The identity provider implicitly relies on the blind signer in order to be able
to authenticate users without forcing them to reveal their identity.

When a user is redirected to her identity provider by a relying party, the
provider checks whether the user has an acess token that has been signed by the
blind signer. The signature on the token may be either publicly verifiable or
privately verifiable. In the former case, the identity provider can verify the
signature on the access token using the blind signer's public key. In the latter
case, the identity provider could send the token to the blind signer and ask
them whether they signed it. The signon process using an access token in the
publicly verifiable case is illustrated in Figure \ref{fig:bss-signon}.

\begin{figure}
  \centering
  \includegraphics[scale=0.6]{figs/fig-bss-signon-color.pdf}
  \caption{Identity Provider Signon with Blind Signed Access Token: (1) IDP asks
  users to authenticate (2) User supplies access token rather than true identity
  or credentials (3) IDP verifies whether BSS signed the token using BSS's
  public key.}
  \label{fig:bss-signon}
\end{figure}

If the access token is valid, the provider is only assured that the user has
been validly authenticated by the blind signer. Thus the provider knows that the
user is a valid user of the blind signer, but does not know which user.

Even in the case of privately verifiable signatures which requires the blind
signer to examine the token, the signer cannot tell who the user is either since
the access token the user presents is not one it has ever seen before. The
signer only ever sees the blinded versions of the access tokens. But the signer
may check that the token has been properly signed and conclude that the user is
valid.

%Deprecated figure, too complex, replaced by two smaller figures
\begin{comment}
\begin{figure}
  \centering
  \includegraphics[scale=0.6]{figs/fig-bss-color.pdf}
  \caption{The user obtains an untraceable access token from the blind signature
  service piror to logging into a website. When his identity provider asks him
  to authenticate, he provides the access token instead of a username or
  password. The identity provider, in turn, verifies that the BSS signed the
  token. If the token has been validly signed, the provider authenticates the
  user to the website he is trying to visit.}
  \label{fig:bss}
\end{figure}
\end{comment}

The access token may be imbued with certain semantics such as an embedded
pseudonym. Upon verying that the access token is legitimate, the identity
provider may extract the pseudonym from the token and assert its validity to the
relying party. Finally, the user is able to login to the website under this
pseudonym.

\subsection{OpenID with PseudoID}

To actually implement PseudoID in a web-based model such as for OpenID, we
consider the user's browser to be acting on the user's behalf during the
protocol. As such, a user's access tokens are stored as cookies in her browser.

Further the blind service is represented as a website with embedded JavaScript
to perform blinding and unblinding functions entirely on the client-side, i.e.
on the user's browser. The blind signer's server only ever sees the blinded
token. The JavaScript code also sets the generated access token as a cookie oni
the identity provider's domain.

Since the identity provider can read cookies on its own domain, it will be able
to check the user's access tokens when authenticating the user. This requires
very little change in the behavior of most OpenID providers which now need to
simply check cookies rather than passwords.

Moreover, private identity providers, i.e. those that implement PseudoID, are
completely compatible with existing OpenID-enabled sites. These relying parties
do not have to change anything about their current OpenID flow in order to be
able to accept users from private identity providers. From the perspective of
the relying party, a private identity provider is indistinguishable from a
regular provider. A private provider simply uses a different authentication
mechanism than most other identity providers, but it still participates in
the same federated login flow outlined in Figure \ref{fig:fedlog}.

\subsection{Caveats}

%System Caveats: Potential for traffic analysis. Need to manage tokens.

%Implementation caveats: Cookies are weak. Need to tunnel through TOR
%and scrub browser. Running crypto in Javascript is unsafe; need to
%trust code.

PseudoID stores a user's access tokens as cookies in the browser. This is not
the ideal solution, but it is done as more of a convenience since cookies are
the most ubiquitous method of persistently storing information in the browser.

Nonetheless, cookies are still an ephemeral data store. By default, they are set
to expire at the end of the current browser session. They also don't follow a
user across his potentially multiple computers with multiple browsers. As
such, it is difficult to extend the user's anonymous identity across different
machines.

The traditional method of aggregating user data on a server to be easily
accessible through multiple clients is not an option if we are to maintain
provable anonymity. Trusting any third-party with logs of a user's accesses
leaves that party susceptible to being subpoenaed for the records and the user's
privacy at their mercy.

Additionally, there are alternate methods of tracking users even if user
accesses are made anonymous to prevent logging. User requests can still be tied
to a particular IP address. But this method of tracking is fragile since users
may be using proxies or otherwise hidden behind network addres translators
(NATs) \cite{Pool00}. Even this method may be avoided if the user uses anonymous
browsing technology like Tor \cite{Tor}.

But in practice, cookies are more often used to track users than IP Addresses.
To ensure that the BSS is not maliciously tagging the user with a uniquely
identifying cookie when the user logs into to create their anonymous access
token, it may be necessary to scrub all cookies not associated with PseudoID
after visiting the blind signer.

Furthermore, the JavaScript code that peforms the blinding and unblinding
operations on the BSS must come from a trusted source.

\subsection{Performace and Cost}

Get latency costs using firebug. Blast GAE with requests. How many to
exhaust the quota? Use that to extrapolate the cost of running a blind
signer.

(This might be an uninteresting section.)

\section{Practical Challenges}

There are several practical considerations dealing with implementation issues as
well as usability.

\subsection{Cryptography in the Browser}

Cryptographic computations often rely on number theoretic manipulations of big
integers, i.e. infinite precision integers, not just normal 32-bit integers. We
used the JSBN library to handle big integer computations in Javascript
\cite{Jsbn}.

\subsection{Cross-Domain Communcation}

Another main challenge was communicating cryptographic tokens across domains
subject to the constraints of the \emph{same-origin policy}. It defines two
pages to have the ``same'' domain if they originate from the same protocol,
host, and port. Thus, for example, a page on the blind signer's domain cannot
set a cookie on or make an AJAX call to the private identity provider's domain.

There are many popular techniques for skirting the same-origin policy such as
using fragment identifiers \cite{Adi07}, nested frames \cite{JaWa07}, or cross-
domain scripts to pass messages between domains.

While the same-origin policy prevents unintentional cross-domain communication
between non-cooperative domains, it still permits communications between
cooperative domains. And in PseudoID, the blind signer and the private identity
provider are indeed cooperative entities.

Thus one simple way of exchanging information between the blind signer and the
identity provider is to load a hidden cross-domain iframe. This technique was
used to allow the blind signer to set a cookie on the identity provider's
domain.

When the user generates his access token, he is on the blind signer's domain. By
the same-origin policy, this access token could not be set directly as a cookie
on the identity provider's domain. But the blind signer can source an iframe to
the identity provider and pass a message in the source URL's fragment
identifier. The identity provider can extract the message from the fragment
identifier. Thus the blind signer may send the cookie as a message that the
identity provider may then set on its own domain.

\section{Extensions and Future Work}

Attribute exchange. Using zero-knowledge proofs instead of Chaum's
blind signer to enable more rich and efficient selective
disclosure.

Cookie same origin policy is a hinderence. We want the ability for our
browser to set a value that is only readable by another domain. Need
plug-in or HTML5 feature?

Changing flow of OpenID: User can pass token to RP, they can verify
directly with IDP. Neither RP nor IDP learn the user's actual
identity. Could be more efficient and usable than OpenID (plus lays
groundwork for Ben and Moti's patent.)

Just citing everything to debug the bib file~
\cite{Cha82,Cha85,GMR89,Json,PyOpenId,ProviderApp,Jsbn,JaWa07,Adi07}.

%Stuff from the Proposal and earlier drafts
\begin{comment}
\section{Privacy Issues}

An evil regime can sieze an IDP's records. And some governments (e.g. European
ones) have privacy requirements of not maintaining logs.

\section{The System}
The system consists of a user with a client application, an anonymized OpenID
Provider (IDP) and one or more relying parties (RPs) that the user wants to log
on to. The RP will interact with the user by asking for an \emph{access token}
and possibly several \emph{attribute token}s that attest to a particular
attribute of the user (e.g. is over 21, is male, lives in the U.S.).

The user has an account at the IDP and logs in normally (with a username and
password) to access the IDP. Once he's logged on to the IDP, he can access its
various services: a blind signature service and an attribute authentication
service. The first service can be used to generate an unlinkable access token
with which the user can log on to the RP. The second service can be used to
generate attribute tokens that the user can use to gain access to various
services at the RP that require them while ensuring the IDP cannot track what
websites he visits.

\subsection{The Client Application}

The client application is the user's interface to his digital identity. It may
be built on top of the browser or as a browser plug-in. It is responsible for
storing the user's various cryptographic keys and mediating requests from the
RP for certain attributes and interacting with the IDP as necessary on the
user's behalf.

For example, suppose the RP requests proof that the user is over 21. The client
application will intercept this request, create a token attesting to the
appropriate attribute without revealing extra information (e.g. birthdate),
handle the blind signature protocol with the IDP to get the token properly
signed, and then present the token to the RP. (The client application will
handle all the necessary technical details as outlined in the patent proposal.)
This will all happen ``behind-the-scenes'' so that the user can continue using
the website without interruption.

\subsection{The Blind Signature Service}

The blind signature service implements the ``cut-and-choose'' protocol whereby
the user sends the signer a set of $n$ blinded tokens $t_1, \ldots, t_n$ to
sign. The signer randomly selects a $j \in [1, \ldots, n]$ and asks the user to
unblind all $t_i$ for $i \neq j$. After verifying that the unblinded $t_i$ are
valid, the signer blindly signs $t_j$ and returns the signature to the user. The
user can then unblind the returned signature to obtain a valid signature on an
unblinded token.

The blind signature service exposes the following API:

\begin{verbatim}
num|String Sign(user_id, tok_list, msg_id=None):
# If msg_id = None, first step of protocol: 
# returns an index j.
# If msg_id != None, second step of protocol: 
# look up related transcript, verify if correct, and if so
# return blind signature on index j.
\end{verbatim}

Whenever a user initiates the blind signature protocol with the service, it
opens a transcript of the new message under \verb,msg_id,. In this transcript,
it keeps track of the original blinded tokens, the index $j$ it chose to keep
hidden, and the second list of unblinded tokens. In the second and final step of
the protocol, the service uses the transcript to check the necessary properties
of the unblinded tokens and ensure that the user is not cheating.

\subsection{The Attribute Authentication Service}

The IDP's attribute authentication service allows the user to generate signed
attribute tokens that he can later present to the RP. The RP would present such
attribute tokens to the IDP for verification.

The service exposes the following API:

\begin{verbatim}
String Validate(user_id, attr)
\end{verbatim}

which uses the blind signature service in a subprocedure to produce a signed
attribute token.

\section{Architecture \& Implementation}

The client application could potentially be implemented as a browser plugin or a
stand-alone desktop application. I need to investigate these options in more
depth, but I am more interested in designing a usable interface for such an
application. I think having an intuitive, simple digital identity manager would
be essential for users to become comfortable with the federated login
experience.

I've experimented with using the sample OpenID provider running on Google
AppEngine in Python \cite{ProviderApp} and a reference consumer (RP) provided by
the Python openid module \cite{PyOpenId}. I've been able to get these modules
set up and hope to tweak the sample AppEngine implementation to fit my needs.

Alternatively, I've considered that it might be easier to do much of the
cryptography with Java, so it may make more sense to build an AppEngine
application in Java from scratch.

I can use JSON \cite{Json} to send and receive data from the various service API
calls (e.g. \verb,Sign,). For example, the \verb,tok_list, parameter could be
sent as a JSON list: \verb|{[t1, ..., tn]}| while the signer's first reply could
be packaged as:

\verb|{"msg_id":"ABC", "index":j}|.
\end{comment}
\bibliography{pseudoid}
\bibliographystyle{acm}

\end{document}

% LocalWords:  login OpenID username
